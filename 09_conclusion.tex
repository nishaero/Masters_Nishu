% !TEX root       = ./type_name.tex
% !TEX program    = pdflatex
% !TEX encoding   = utf-8
% !TEX spellcheck = de_DE_frami
%=======================================================================

\chapter{Conclusion}\label{ch:Conclusion}

OpenStack is an emerging and stable open source Cloud Infrastructure-as-a-Service operating solution.
OpenStack, with it's usage as Compute service- to create cloud based computing or Storage service- for cloud based storage solutions or both combined, is an easily deployable cloud operating solution with minimum infrastructure to start hosting the Cloud service.

As this thesis is focussed on scheduling algorithm of requested virtual instances, it can be observed from the logs in Appendix \ref{app:sec:filterschedulerlogtrace10vi} that, with the standard nova FilterScheduler, when a large number of virtual instance creation is requested, it iterates for that number of times to provide a placement decision one after the another. There is an opportunity for improvement to solve a large requests with minimum time and better mathematical modelling.

In the section \nameref{sec:comparision} of the chapter \nameref{ch:comparisionofbothscheduler}, it can be observed that the cPlex based scheduling algorithm is effective in providing a placement decision which would reduce the operating power expense of the OpenStack cloud cluster by minimising the active(powered on) number of hosts.

This is also effective in timely creation of large quantity of virtual instances compared to the existing standard nova scheduler.

This thesis is an approach to have a different possible solution in the scheduling algorithm which could be helpful to have constraint dependant scheduling.

This thesis can also be extended to have two different scheduling solutions depending on the number of requests for creation of the virtual instances.
With a condition based on quantity of instance creation, the selection of the scheduler driver can be switched at runtime.

If the "Live Migration" could work effectively, the scheduling algorithm could also include the migration of existing instances to re-arrange itself which would increase the available capacity on the hosts and place new instances with larger configuration requirements.
This could also be helpful for live migration of multiple virtual instances when the compute nodes needs a maintenance downtime.

The cloud computing is the growing field of interest which creates lots of opportunities for research and as well as the ideas for business.