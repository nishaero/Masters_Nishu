% !TEX root = ..</vorlage.tex
%=======================================================================

%== Glossaries, Nomenclature, Lists of Symbols and Acronyms
% http://www.latex-community.org/index.php?option=com_content&view=article&id=263:glossaries-nomenclature-lists-of-symbols-and-acronyms&catid=55:latex-general&Itemid=114
%== Acronym und Glossary mit gleichen Einträgen
% http://ewus.de/tipp-1029.html
%== Making acronym entries link to glossary entries
% http://www.latex-community.org/forum/viewtopic.php?f=4&t=2976

%===================
% nur glossarie
%===================

\newglossaryentry{gls:app}{name={Application},
text={App},
description={dt. Anwendung, Schlagwort für eine kleine Anwendung, welches vorrangig im Bereich mobiler, eingebetteter Systeme Verwendung findet.},
first={App (vom engl. Application)}
}

%===================
% glossarie + acronym
%===================
\newacronym{ic}{IC\glsadd{gls:ic}}{Integrated Circuit}
\newglossaryentry{gls:ic}{%
	name={Integrated Circuit},
	description={dt. integrierter Schaltkreis; Eine auf einem einzelnen Silizium-Die integrierte Schaltung},
}

\newacronym{gui}{GUI}{Graphical User Interface}
\newglossaryentry{gls:gui}{%
	name={Graphical User Interface},
	description={dt. Grafische Benutzerschnittstelle; Eine Bedienoberfläche, die dem Benutzer die Interaktion mit einem System über grafische Symbole ermöglicht},
}

%===================
% nur acronym
%===================
\newacronym{fpga}{FPGA}{Field Programmable Gate Array}
\newacronym{sw}{SW}{Software}
\newacronym{hw}{HW}{Hardware}

% vim: ts=2:sw=2:noexpandtab:fileformat=unix
