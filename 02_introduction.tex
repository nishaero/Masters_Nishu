% !TEX root       = ./type_name.tex
% !TEX program    = pdflatex
% !TEX encoding   = utf-8
% !TeX spellcheck = en_GB
%=======================================================================

\chapter{Introduction}\label{ch:introduction}
\sffamily{}
The penetration of mobile internet users has increased many fold from a few thousands to millions over a short span of time. Due to the increasing demand for data among subscribers, mobile operators are pushed to go beyond boundaries to provide efficient and reliable data service to their customers. Although, the existing network services such as UMTS and LTE can handle larger data capacity, their coverage is not always sufficient in crowded places such as office buildings, convention centers, shopping malls etc. There is an urgent need to find a solution on how to offload the mobile data traffic over to other radio standards.

In such case, WLAN is an existing radio standard that has already been deployed in large numbers and has been supported by millions of devices lately. One unique advantage of using WLAN over other radio standards would be its license free usage of its radio frequency for commercial purposes. This WLAN standard, when deployed in a controlled manner can support data traffic routed from the mobile services. The IEEE 802.11 WLAN has already been widely used for commercial enterprises ranging from office networks, shopping malls to educational institutions etc. The deployment rages from a few dozens to hundreds of access points (APs), which serve many users through multitude of devices ranging from mobile devices, laptops to printers and other connected hardware. These networks also provide varied set of services that includes authentication, authorization and accounting (AAA), dynamic channel reconfiguration, interference management, security such as intrusion detection and prevention and providing quality of service. 

These enterprise WLAN AP’s are usually centrally managed through a controller. The task now is to find a solution to seamlessly direct traffic between LTE and WLAN. The research project BIC-IRAP (Business Indoor Coverage Integrated Radio Access Point) is currently aimed at providing a solution for the seamless coupling between LTE and WLAN.

The growing adoption of Software Defined Networking in the recent years has given rise to providing unique solutions without depending too much on hardware. The advantage of using SDN is that, it separates the network control pane from the physical network topology and uses software control flow to define how traffic is forwarded in the network. For example, the routing table and the flow control of a switch can be easily controlled remotely through a software controller.  The capabilities of SDN is possible due to the use of OpenFlow protocol which is a standardized protocol that is used by the SDN based controller to manipulate the flow tables of network switches. This provides more flexibility to programmatically control the behavior of network switches by building network applications that talk to the network controller. Any OpenFlow enabled switch from any vendor provides a common interface to be manipulated via a controller, thus providing flexibility and simplified network management.

%\comment{
%The cloud is a big deal for given reasons:
%\begin{itemize}
%	\item It does not need any effort on the consumer or the end user's part to maintain or manage it.
%	\item Any authenticated user can access the cloud-based applications and services from anywhere. All that the user needs is a device with an Internet connection.
%	\item It is effectively infinite in size, so that the consumer or the end user doesn't need to worry about it running out of capacity.
%	\item The resources can be scaled up or down quickly and easily to meet the desired changing demands of the consumer.
%	\item The services offered in the cloud are metered services, so the end user pay only for what they use.
%\end{itemize}
%}

\section{Contribution}\label{sec:contribution}

This thesis provides a novel approach towards separating the data traffic between the different network providers within an access point. This is made possible through the simple, yet effective use of OpenFlow protocol that enables the development of different enterprise WLAN services as applications such as, using software defined network controllers. The performance benefits achieved though this system is possible without any changes to the existing 802.11 client. The proposed system is compatible with the existing enterprise WLAN security protocols like WPA2 enterprise.

\section{Results}\label{sec:Results}
The expected outcome of this thesis is to demonstrate a prototype system that runs an AP on an OpenFLow controlled switch. The AP also provides enterprise grade authentication system WPA2 enterprise alongside a RADIUS server, and host two separate networks. The separation of data traffic between the two networks will take place within the same AP and provide Hotspot 2.0 functionality.

\section{Research Context\cite{BIC:IRAP}}\label{sec:BIC-IRAP}
The research described in this thesis was done based on the BIC-IRAP project which is focused on combining the strengths of LTE and Wireless-LAN seamlessly. Through the integration of small and micro cells of LTE with WLAN in the BIC-IRAP system, the two radio technologies are available through a single dynamically configurable hardware configuration.

\section{Thesis Structure}\label{sec:Structure}
This thesis report is organized as follows, Chapter 2 describes the background for this thesis. Chapter 3 describes in detail about SDN and the different types in use today. Chapter 4 talks about the control and authentication mechanism such as the protocols and technologies used. Chapter 5 talks about the environment required to build the system such as the tools and software’s. Chapter 6 describes in detail how the application is developed, from conceptualization to coding in python. Chapter 7 shows how the system is being implemented. Chapter 8 presents the results obtained from the system after series of testing and enhancement. Chapter 9 concludes the thesis and describes further improvements and drawbacks of this method.