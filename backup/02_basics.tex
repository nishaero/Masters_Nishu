% !TEX root       = ./type_name.tex
% !TEX program    = pdflatex
% !TEX encoding   = utf-8
% !TEX spellcheck = de_DE_frami
%=======================================================================

\chapter{Installation and Configuration of OpenStack}\label{ch:config}

\blindtext[1]

\section{Technical requirements}\label{sec:erstegdl}

Weiterführende Informationen zu diesem Thema bietet~\cite{Herrmann2004}. In dem Buch werden auch Details zu \acp{ic}
dargestellt.
\blindtext[2]
\blindenumerate{}
\blindtext[2]

\subsection{Configuration of Services and config files}\label{ssec:ersteUGdl}

Das hier vorgestellte basiert auch dem Buch von~\cite{Heinkel2000}. Die Lösung kann auch mit einem \ac{fpga} sowie
\ac{hw} und \ac{sw} entstehen.
\blindtext[4]

%\subsection{Zweite Untergrundlage}\label{ssec:zweiteUGdl}

\blindtext[1]
\begin{table}[htpb]
  \centering
  \caption{Eine tolle Tabelle}\label{tab:ett}
  \begin{tabular}{lcrl}
  Links & Zentriert & Rechts & Links \\
  \toprule
  eins & zwei & drei & vier \\
  eins & zwei & drei & vier \\
  eins & zwei & drei & vier \\
  \end{tabular}
\end{table}

Eine Tabelle benötigt auch eine Referenz, wie diese hier zu \ref{tab:ett}.
\blindtext[1]

%\section{Zweite Grundlage}\label{sec:zweitegdl}

%\blindtext[2]

%\section{Dritte Grundlage}\label{sec:drittegdl}

%\blindtext[2]

% vim: ts=2:sw=2:sts=2:expandtab:wrapmargin=2:tw=120

