% !TEX root       = ./type_name.tex
% !TEX program    = pdflatex
% !TEX encoding   = utf-8
% !TeX spellcheck = en_GB
%=======================================================================

\chapter{Introduction to Cloud Computing}\label{ch:introduction_cloud}
\sffamily{}
"Moving to the cloud", "running in the cloud", "stored in the cloud", "accessed from the cloud", these days it is probably seen that most of the activities are happening in "the cloud". If any person would like to share their data, for example digital documents, software source code, software, pictures or any videos, it is normally said to share them over "the cloud". But what exactly is this "The Cloud" and why is this "The Cloud" so important?

The short answer is that, "the cloud" is somewhere at the other end of the user's Internet connection, a place where any user can access applications and services, and where user's data can be stored securely by using any device connected over the Internet. "The cloud" is the delivery of the on-demand request of computing resources, everything and anything from applications, tools, storage, to data centres, over the Internet on a pay-for-use basis.

The Cloud is important because, there appears to be a shift going on from office-based work to working on the move. Where in the past, people would run applications or programs from software downloaded on a physical computer or server in their building, cloud computing allows people to access the same kinds of applications over the internet. The convenience of the hand held devices enabled with high speed Internet, enables the people to bridge the gap of accessing either data stored remotely or run any system or machine remotely with ease.

\comment{
The cloud is a big deal for given reasons:
\begin{itemize}
	\item It does not need any effort on the consumer or the end user's part to maintain or manage it.
	\item Any authenticated user can access the cloud-based applications and services from anywhere. All that the user needs is a device with an Internet connection.
	\item It is effectively infinite in size, so that the consumer or the end user doesn't need to worry about it running out of capacity.
	\item The resources can be scaled up or down quickly and easily to meet the desired changing demands of the consumer.
	\item The services offered in the cloud are metered services, so the end user pay only for what they use.
\end{itemize}
}

\section{Where did the cloud come from?\cite{salesforce} }\label{sec:history}

The Internet has its roots in the 1960's. But not until the early 1990's that it had any relevance for businesses. The World Wide Web (WWW) was born in 1991, and in 1993 a web browser called Mosaic was released that allowed users to view web pages that included graphics as well as text. This heralded the first company web sites – and not surprisingly, most of these belonged to companies involved in computing and technology.

As Internet connections got faster and more reliable, a new type of company called an "Application Service Provision" or ASP started to appear. ASPs took the existing business applications and ran them for their customers. The ASP would buy the computing hardware and keeping the application running, and the customer would pay a monthly fee to access it over the Internet.

But it wasn't until right at the end of the 1990's that cloud computing as it is known today did appear. That is when a company by name "salesforce.com" in 1999 introduced its own multi-tenant application which was specifically designed:
\begin{itemize}
	\item to run "in the cloud";
	\item to be accessed over the Internet from a web browser;
	\item to be used by large numbers of customers simultaneously at low cost.
\end{itemize}
Since then the cloud has enormously grown and is still growing.

\section{Services of "The Cloud"\cite{IBMcloud}}\label{sec:service_models}
The Cloud is a very broad concept, and it covers just about every possible sort of online service. Anything as a service normally abbreviated as XaaS, provided remotely can be generalised as a cloud service. But when businesses refer to \textbf{"cloud procurement"}, there are usually three models of cloud services under consideration:
\begin{itemize}
	\item \textbf{Software as a Service (SaaS)}
	\item \textbf{Platform as a Service (PaaS)}
	\item \textbf{and Infrastructure as a Service (IaaS)}
\end{itemize}

The cloud infrastructure as a service model can also be segregated based on the type of cloud network desired by the business which can be:
\begin{itemize}
	\item Private cloud: with a cloud network accessible by limited and restricted permission to only particular organisation.
	\item Public cloud: with a cloud network accessible to the general public.
	\item and Hybrid cloud: which is partly private and secured, and partly public and accessible.
\end{itemize}

\begin{figure}[H]
	\begin{center}
		\includegraphics[width=1\linewidth]{responsibilities}
		\caption{Separation of end user responsibilities based on Internet cloud service model}
		\label{fig:responsibilities}
	\end{center}
	\vspace{-10pt}
\end{figure}

The figure \ref{fig:responsibilities} about the cloud service model describes what would be the roles that needs to be played by an end user to utilise and manage each type of service model compared to working on on-premise systems.

\subsection{Software as a Service (SaaS)}\label{ssec:SaaS}
Software as a service (SaaS) is a software distribution model in which a third-party provider hosts the applications and makes them available to customers over the Internet. Cloud-based applications or software as a service, run on distant computers "in the cloud" that are owned and operated by others and that connect to the user's computer via the Internet and, usually, a web browser.

\begin{figure}[H]
	\begin{center}
		\includegraphics[width=0.95\linewidth]{cc_SaaS}
		\caption{Software as a service model\cite{IBMcloud:SaaS}}
		\label{fig:cc_SaaS}
	\end{center}
	\vspace{-10pt}
\end{figure}

SaaS is the most familiar form of cloud service for consumers. SaaS moves the task of managing software and its deployment to third-party services. Among the most familiar SaaS applications for business are customer relationship management applications like Salesforce, productivity software suites like Google Apps, and storage solutions like Box and Dropbox.

Use of SaaS applications tends to reduce the cost of software ownership by removing the need for technical staff to manage the installation, maintenance, and upgrade software, as well as reduce the cost of licensing software. SaaS applications are usually provided on a subscription model.

To summarise the benefits of SaaS:
\begin{itemize}
	\item The end user can sign up and rapidly start using innovative business apps
	\item Apps and data are accessible from any device connected to Internet
	\item No data is lost if the user's computer or a device breaks, as the data is in the cloud
	\item The service is able to dynamically scale to the usage needs
\end{itemize}

\subsection{Platform as a Service (PaaS)}\label{ssec:PaaS}
Platform as a service (PaaS) is a cloud computing model that delivers development and management tools or applications over the Internet. In a PaaS model, a cloud provider delivers hardware and software tools, "usually those needed for application development", to its users as a service. The provider of the PaaS hosts the hardware and software on their own infrastructure. As a result, PaaS frees the users from having to install in-house hardware and software to develop and run a new application.

It provides a cloud-based environment with everything required to support the complete lifecycle of building and delivering web-based (cloud) applications, without the cost and complexity of buying and managing the underlying hardware, software, provisioning, and hosting. The PaaS provider on request can provide a web development tool or platform which can be customised for the end user.


\begin{figure}[H]
	\begin{center}
		\includegraphics[width=0.95\linewidth]{cc_PaaS}
		\caption{Platform as a service model\cite{IBMcloud:PaaS}}
		\label{fig:cc_PaaS}
	\end{center}
	\vspace{-10pt}
\end{figure}

For example, deploying a typical business tool locally might require an IT team to buy and install hardware, operating systems, middleware (such as databases, Web servers and so on) the actual application, define user access or security, and then add the application to existing systems management or application performance monitoring (APM) tools. IT teams must then maintain all of these resources over time. A PaaS provider, however, supports all the underlying computing and software; users only need to log in and start using the platform – usually through a Web browser interface.

Common PaaS vendors include Salesforce.com's Force.com, which provides an enterprise customer relationship management (CRM) platform. PaaS platforms for software development and management include Appear IQ, Mendix, Amazon Web Services (AWS) Elastic Beanstalk, Google App Engine and Heroku.

To summarise the benefits of PaaS:
\begin{itemize}
	\item Develop applications and get to market faster
	\item Deploy new web applications to the cloud in minutes
	\item Reduce complexity with middleware as a service
\end{itemize}

\subsection{Infrastructure as a Service (IaaS)}\label{ssec:IaaS}

Infrastructure as a Service (IaaS) is a form of cloud computing that provides virtualized computing resources over the Internet. Infrastructure as a service provides consumers with computing resources including servers, networking, storage, and data center space on a pay-per-use basis.

\begin{figure}[H]
	\begin{center}
		\includegraphics[width=0.95\linewidth]{cc_IaaS}
		\caption{Infrastructure as a service model\cite{IBMcloud:IaaS}}
		\label{fig:cc_IaaS}
	\end{center}
	\vspace{-10pt}
\end{figure}

In an IaaS model, a third-party provider hosts hardware, software, servers, storage and other infrastructure components on behalf of its users. IaaS providers also host users' applications and handle tasks including system maintenance, backup and resiliency planning.

IaaS platforms offer highly scalable resources that can be adjusted on-demand. This makes IaaS well-suited for workloads that are temporary, experimental or change unexpectedly.

Other characteristics of IaaS environments include the automation of administrative tasks, dynamic scaling, desktop virtualization and policy-based services.

For example, if a business is developing a new software product, it might be more cost-effective to host and test the application through an IaaS provider. Once the new software is tested and refined, it can be removed from the IaaS environment for a more traditional in-house deployment or to save money or free the resources for other projects.

Leading IaaS providers include Amazon Web Services (AWS), Windows Azure, Google Compute Engine, Rackspace Open Cloud, IBM SmartCloud Enterprise and ProfitBricks.
To summarise the benefits of IaaS:
\begin{itemize}
	\item No need to invest in having or owning the hardware
	\item Infrastructure scales on demand to support dynamic workloads
	\item Flexible, innovative services available on demand
\end{itemize}

\section{Summary about "The Cloud"}\label{sec:intro1_summary}
The cloud has become a key enabler for today's small or medium sized business to unlock creativity, drive unrivaled innovation and level the competition with larger enterprise competitors.

Prior to the advent of cloud-based products software solutions delivered over the Internet companies were often forced to invest in servers and other products to run software and store data. The advent of cloud services as well as their steady improvement in such areas as security and reliability make these solutions a logical choice for business owners and principals who want the latest innovations, functionality, and efficiency as well as cost effectiveness.

Many businesses garner considerable cost savings by migrating their software systems to the cloud. In addition to reducing reliance on the purchase and maintenance of servers, companies often lower their information technology costs in such areas as dedicated personnel and software upgrades. Most cloud services upgrade and update software via the Internet with little or no downtime for end users, decreasing the wait time associated with installing and testing software on an on-site network. Moreover, cloud-based products are scalable: unlike conventional software, cloud services can be expanded as needed to encompass as many end users as required without additional overhead of servers and upgrades to handle added workloads.

When more number of small and medium sized business begin to realize that the cloud can do more than just reduce the cost of IT, the journey to the cloud becomes inevitable and the question shifts from whether to adopt cloud technologies to how to do so sensibly.

It is fascinating to understand the working of cloud as it has created lot of new opportunities for research, jobs and investment in this domain.