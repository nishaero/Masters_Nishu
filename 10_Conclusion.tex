% !TEX root       = ./type_name.tex
% !TEX program    = pdflatex
% !TEX encoding   = utf-8
% !TEX spellcheck = de_DE_frami
%=======================================================================
\chapter{Conclusion}
The software defined networking approach to orchestrating networks is picking up momentum. By using open interfaces and protocols to programmatically control network elements, it is becoming easier to introduce new functionalities to a network, without being constrained by vendor lock-in. In this thesis, we consider the specific case of introducing user segregation to enterprise WLANs. Through the user segregation application, we have demonstrated that by using a set of abstractions, it is possible to segregate users based on their de-vice mac address and host two networks within an Access Point and provide isolation between them.

\section{Discussion}
This thesis explores the idea of hosting flexible enterprise WLAN networks within an single Access Point, wherein the access points being managed by a software defined network based controller and the isolating the users and their networks at the MAC level. We have taken a step forward in achieving user segregation as outlined in the chapter 6 \nameref{ch:application_design}. The ideas described are validated through an implementation of the application described in chapter 7 \nameref{ch:implementation}. A performance evaluation of the system as described in chapter 8 \nameref{ch:Evaluation}, which demonstrates the practicality of the system and the application within the context of the test environment at the University’s communication networks lab. The tests were conducted using mobile devices and PC’s in a controlled environment with predefined users to simulate an actual network.

\section{Technology Demonstrator}
The results obtained from the tests conducted were satisfactory. The application performed as designed and could achieve isolation of users between networks and the modification of flow table in the Open vSwitch happened in real time for users with assigned out port ids from the database. 

The application was designed to test the capability of the openflow based controller and OpenWrt to host and manage multiple networks within an access point and provide authentication based on RADIUS similar to Hotspot 2.0 to provide seamless mobility for users when roaming on different networks. There were many constrains faced during the designing of this application, the OpenWrt system doesn’t support implementation of Hotspot2.0 in its system, memory constraints and the processor speeds also impacted the performance of the rotuer as a result the Freeradius and MySQL has to be installed in a separate machine, instead on the access point itself. One of the major reasons that this application cannot be used in a real environment is a serious lack of security. The application only provides secure connection between the users and their corresponding IIS servers while the other network and its iis server is completely isolated. This can be seen the figure. Whereas the IIS servers can communicate with each outer and also due to ARP request from the servers with the switch, the OVS learns the path to all the connected devices and thus allowing one server to access the mobile device connected in the other network in reverse. This compromises the security when one server is breached, then it can be used to gain access to all the devices connect to the network.

\section{Future Enhancements}
The application and the system can be further enhanced by introducing user groups on each network and make the SDN controller to behave as a firewall and provide access control to these groups thereby enhancing the security and avoiding full network access during security breach. The other possible extension would be to use a different device as an access point which has more memory and computing power. This can allow integrating Freeradius and MySQL within the access point and thereby making the system more independent and controlled only by the SDN controller. This would ultimately simplify installation and reduce deployment costs overall.