% !TEX root       = ./type_name.tex
% !TEX program    = pdflatex
% !TEX encoding   = utf-8
% !TEX spellcheck = de_DE_frami
%=======================================================================

\chapter{User Segregation Application code}\label{app:ch:app_code}

In this Appendix \nameref{app:ch:app_code}, the entire application code written in Python is listed here.


\begin{lstlisting}[language = Python, caption={The User Segregation Mac Learning Application}, label={lst:userseg-code}]
# Copyright (C) 2011 Nippon Telegraph and Telephone Corporation.
#
# Licensed under the Apache License, Version 2.0 (the "License");
# you may not use this file except in compliance with the License.
# You may obtain a copy of the License at
#
#    http://www.apache.org/licenses/LICENSE-2.0
#
# Unless required by applicable law or agreed to in writing, software
# distributed under the License is distributed on an "AS IS" BASIS,
# WITHOUT WARRANTIES OR CONDITIONS OF ANY KIND, either express or
# implied.
# See the License for the specific language governing permissions and
# limitations under the License.
import MySQLdb
import sys
import datetime
from ryu.base import app_manager
from ryu.controller import ofp_event
from ryu.controller.handler import CONFIG_DISPATCHER, MAIN_DISPATCHER
from ryu.controller.handler import set_ev_cls
from ryu.ofproto import ofproto_v1_3
from ryu.lib.packet import packet
from ryu.lib.packet import ethernet
from ryu.lib.packet import ether_types
from ryu.lib.packet import udp



class SimpleSwitch13(app_manager.RyuApp):
	OFP_VERSIONS = [ofproto_v1_3.OFP_VERSION]

	def __init__(self, *args, **kwargs):
		super(SimpleSwitch13, self).__init__(*args, **kwargs)
		self.mac_to_port = {}

	@set_ev_cls(ofp_event.EventOFPSwitchFeatures, CONFIG_DISPATCHER)
	def switch_features_handler(self, ev):
		datapath = ev.msg.datapath
		ofproto = datapath.ofproto
		parser = datapath.ofproto_parser

		# install table-miss flow entry
		#
		# We specify NO BUFFER to max_len of the output action due to
		# OVS bug. At this moment, if we specify a lesser number, e.g.,
		# 128, OVS will send Packet-In with invalid buffer_id and
		# truncated packet data. In that case, we cannot output packets
		# correctly.  The bug has been fixed in OVS v2.1.0.
		match = parser.OFPMatch()
		actions = [parser.OFPActionOutput(ofproto.OFPP_CONTROLLER,
		ofproto.OFPCML_NO_BUFFER)]
		self.add_flow(datapath, 0, match, actions)

	def add_flow(self, datapath, priority, match, actions, buffer_id=None):
		ofproto = datapath.ofproto
		parser = datapath.ofproto_parser
		
		inst = [parser.OFPInstructionActions(ofproto.OFPIT_APPLY_ACTIONS, actions)]
		if buffer_id: 
			mod = parser.OFPFlowMod(datapath=datapath, buffer_id=buffer_id, priority=priority, match=match, instructions=inst)
		else:
			mod = parser.OFPFlowMod(datapath=datapath, priority=priority, match=match, instructions=inst)
		datapath.send_msg(mod)

	@set_ev_cls(ofp_event.EventOFPPacketIn, MAIN_DISPATCHER)
	def _packet_in_handler(self, ev):
		# If you hit this you might want to increase
		# the "miss_send_length" of your switch
		if ev.msg.msg_len < ev.msg.total_len:
			self.logger.debug("packet truncated: only %s of %s bytes", ev.msg.msg_len, ev.msg.total_len)
		msg = ev.msg
		datapath = msg.datapath
		ofproto = datapath.ofproto
		parser = datapath.ofproto_parser
		in_port = msg.match['in_port']
		
		pkt = packet.Packet(msg.data)
		eth = pkt.get_protocols(ethernet.ethernet)[0]
		udp_payload = pkt.get_protocols(udp.udp)
		
		
		if eth.ethertype == ether_types.ETH_TYPE_LLDP:
			# ignore lldp packet
			return
		dst = eth.dst
		src = eth.src
		
		dpid = datapath.id
		self.mac_to_port.setdefault(dpid, {})
	#self.logger.info ("Destination is %s", eth.dst)	

	#creating a mysql connection to database -last edit 17/11

	connection = MySQLdb.connect(host = "192.168.1.169", user = "freerad", passwd = "pass", db = "radius")
	cursor = connection.cursor ()

	#cursor.execute ("select CallingStationId from radpostauth order by id desc LIMIT 1")
	#data = cursor.fetchall ()
	cursor.execute ("SELECT portid FROM radcheck WHERE username IN (SELECT user FROM radpostauth WHERE CallingStationId = %s AND id = (SELECT MAX(id) from radpostauth) )", src)
	outport_for_src = cursor.fetchone ()
	self.logger.info ("outport_for_src tuple is %s", outport_for_src)
	#portid = outport_for_src[0]
	self.logger.info ("Data is %s", src)	
	cursor.close()
	connection.close ()
	# Mysql verification end

		self.logger.info("packet in %s %s %s %s", dpid, src, dst, in_port)


		# learn a mac address to avoid FLOOD next time.
		self.mac_to_port[dpid][src] = in_port

	#self.logger.info ("Mac to port in dst is %s ",self.mac_to_port[dpid] )
	self.logger.info ("DST is %s", dst)
		if dst in self.mac_to_port[dpid]:
		test = self.mac_to_port[dpid][dst]
		self.logger.info ("self.mac_to_port[dpid][dst] value %s", test)
		self.logger.info ("Out_Port before condition %s", outport_for_src)
		#try:		
		#if portid in self.mac_to_port[dpid][dst]:
		if outport_for_src != None and all(outport_for_src):
			if int(outport_for_src[0]) == self.mac_to_port[dpid][dst]:
				out_port = self.mac_to_port[dpid][dst]
				self.logger.info ("Out_Port is the same as in database table %s", out_port)
			else:
				self.logger.info ("Out_Port is not allowed, dropping packet")	
				return

		else:
		#except (TypeError,UnboundLocalError):		
			out_port = self.mac_to_port[dpid][dst]		
			self.logger.info ("Out_Port not defined in database, using learned port")	
			#return
		else:
		#try:
		#if portid > 0 :
		self.logger.info ("Out_Port before else flood condition %s", outport_for_src)
		if outport_for_src != None and all(outport_for_src):
			out_port = int(outport_for_src[0])
			self.logger.info ("Setting Out_Port same as in table, changing flow %s", out_port)
		else:
		#except (TypeError,UnboundLocalError):

			out_port = ofproto.OFPP_FLOOD
		#out_port = 4
			self.logger.info ("Out_Port is Flooded %s", out_port)

	self.logger.info ("Above actions Out_Port %s", out_port)
		actions = [parser.OFPActionOutput(out_port)]
	self.logger.info ("Actions is %s", actions)

		# install a flow to avoid packet_in next time
		if out_port != ofproto.OFPP_FLOOD:
		self.logger.info ("Out_Port not flooded adding flow ")	
			match = parser.OFPMatch(in_port=in_port, eth_dst=dst)
			# verify if we have a valid buffer_id, if yes avoid to send both
			# flow_mod & packet_out
			if msg.buffer_id != ofproto.OFP_NO_BUFFER:
				self.add_flow(datapath, 1, match, actions, msg.buffer_id)
				return
			else:
				self.add_flow(datapath, 1, match, actions)
		data = None
		if msg.buffer_id == ofproto.OFP_NO_BUFFER:
			data = msg.data

		out = parser.OFPPacketOut(datapath=datapath, buffer_id=msg.buffer_id, in_port=in_port, actions=actions, data=data)
		datapath.send_msg(out)


\end{lstlisting}

