% !TEX root       = TODO ./00_abstract
% !TEX program    = pdflatex
% !TEX encoding   = utf-8
% !TEX spellcheck = de_DE_frami
%=======================================================================
% File          : 00_abstract.tex
% Author(s)     : Daniel Kriesten
% Email         : daniel.kriesten@etit.tu-chemnitz.de
% Creation Date : <Mo 09 Mär 2015 19:57:18 krid>
%=======================================================================

\begin{abstract}
\sffamily{}
\qquad The ever-increasing use of data services on mobile devices, places increased demands on existing networks. Especially in busy areas, such as shopping centers, office buildings or in event centers, the existing network coverage by UMTS and LTE is no longer sufficient. It is, therefore, obvious to direct some traffic through other radio standards. In this case, WLAN is particularly suitable because those frequencies are free to use without any license restrictions and since most mobile devices have long since supported this. However, an uncontrolled number of WLAN access points can interfere with each other. It is, therefore, desirable to install only one set of access points at these locations and manage them centrally. The research project BIC-IRAP (Business Indoor Coverage Integrated Radio Access Points) is a project aimed at providing a seamless coupling between LTE and WLAN.

\qquad The separation of data traffic is an important aspect when using shared hardware.  No direct data exchange between the networks of different mobile radio providers should be possible. Likewise, the networks of different businesses or companies should be kept strictly separate from each other.
Classic VLANs would be used for this purpose. Within the scope of the BIC-IRAP project, however, there were considerations to control parts of the network using SDN. Therefore, the goal of this master thesis is to operate an access point (AP) on an Open Flow-controlled switch. Users can be authenticated against a RADIUS server. The AP should supply at least two separate networks. If possible, the separation of data traffic should already take place in the AP. Optionally the AP should provide Hotspot 2.0 functionality.

\qquad The conceptualization and implementation must be documented in detail. The optional components are carried out in consultation with the supervisor. The successful completion of the work is a test set-up. The achievable performance characteristics must be recorded.

\end{abstract}